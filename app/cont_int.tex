\begin{enumerate}[label=\thechapter.\arabic*,ref=\thechapter.\theenumi]
\numberwithin{equation}{enumi}
\numberwithin{figure}{enumi}
\numberwithin{table}{enumi}
\item 
\begin{align}
    x\brak{n}          & \xrightarrow{\text{Z}} X\brak{z}             \\
    \implies X\brak{z} & = \sum_{k=-\infty}^{\infty} x\brak{k} z^{-k}
\end{align}
Multiplying both side with $z^{k-1}$ and integrating on a contour integral enclosing the region of convergence. Where $C$ is a counter-clockwise closed contour in region of convergence.
\begin{align}
    \frac{1}{2\pi j} \oint_C X\brak{z} z^{k-1}  dz & = \frac{1}{2\pi j} \oint_C \sum_{k=-\infty}^{\infty} x\brak{k} z^{-n+k-1} dz                               \\
                                                   & =\sum_{k=-\infty}^{\infty} x\brak{k}  \frac{1}{2\pi j} \oint_C z^{-n+k-1} .dz \label{eq:cauchyintegraleq1}
\end{align}
From cauchy's integral theorem
\begin{align}
    \frac{1}{2\pi j} \oint_C z^{-k} dz & = \begin{cases}
                                               1, \quad k = 1 \\
                                               0, \quad k\neq 1
                                           \end{cases}  \\
                                       & = \delta\brak{1-k}
\end{align}
So eq \eqref{eq:cauchyintegraleq1} becomes
\begin{align}
    \frac{1}{2\pi j} \oint_C X\brak{z} z^{k-1} dz & = \sum_{k=-\infty}^{\infty} x\brak{k} \delta \brak{k-n}                       \\
    \implies x\brak{n}                            & =  \frac{1}{2\pi j} \oint_C X\brak{z} z^{n-1} dz \label{eq:cauchyintegraleq2}
\end{align}
Contour integrals like \eqref{eq:cauchyintegraleq2} can be evaluated using Cauchy's residue theorem.
\begin{align}
    x\brak{n} & =  \frac{1}{2\pi j} \oint_C X\brak{z} z^{n-1} dz                               \\
              & = \sum \sbrak{\text{Residue of } X\brak{z} z^{n-1} \text{ at poles inside } C}
\end{align}

\end{enumerate}
