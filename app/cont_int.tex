\begin{enumerate}[label=\thechapter.\arabic*,ref=\thechapter.\theenumi]
\numberwithin{equation}{enumi}
\numberwithin{figure}{enumi}
\numberwithin{table}{enumi}
\item 
\begin{align}
    x\brak{n}          & \xrightarrow{\text{Z}} X\brak{z}             \\
    \implies X\brak{z} & = \sum_{k=-\infty}^{\infty} x\brak{k} z^{-k}
\end{align}
Multiplying both side with $z^{k-1}$ and integrating on a contour integral enclosing the region of convergence. Where $C$ is a counter-clockwise closed contour in region of convergence.
\begin{align}
    \frac{1}{2\pi j} \oint_C X\brak{z} z^{k-1}  dz & = \frac{1}{2\pi j} \oint_C \sum_{k=-\infty}^{\infty} x\brak{k} z^{-n+k-1} dz                               \\
                                                   & =\sum_{k=-\infty}^{\infty} x\brak{k}  \frac{1}{2\pi j} \oint_C z^{-n+k-1} .dz \label{eq:cauchyintegraleq1}
\end{align}
From cauchy's integral theorem
\begin{align}
    \frac{1}{2\pi j} \oint_C z^{-k} dz & = \begin{cases}
                                               1, \quad k = 1 \\
                                               0, \quad k\neq 1
                                           \end{cases}  \\
                                       & = \delta\brak{1-k}
\end{align}
So eq \eqref{eq:cauchyintegraleq1} becomes
\begin{align}
    \frac{1}{2\pi j} \oint_C X\brak{z} z^{k-1} dz & = \sum_{k=-\infty}^{\infty} x\brak{k} \delta \brak{k-n}                       \\
    \implies x\brak{n}                            & =  \frac{1}{2\pi j} \oint_C X\brak{z} z^{n-1} dz \label{eq:cauchyintegraleq2}
\end{align}
Contour integrals like \eqref{eq:cauchyintegraleq2} can be evaluated using Cauchy's residue theorem.
\begin{align}
    x\brak{n} & =  \frac{1}{2\pi j} \oint_C X\brak{z} z^{n-1} dz                               \\
              & = \sum \sbrak{\text{Residue of } X\brak{z} z^{n-1} \text{ at poles inside } C}
\end{align}
\item 
Question: Find the sum of n terms of an AP where common difference = $d$ using Contour Integration.\\
\solution\\
By performing inverse Z transform on S\brak{z} using contour integration
\begin{align}
   s(n)&=\frac{1}{2\pi j}\oint_{C}S(z) \;z^{n-1} \;dz  \\
   s(n)&=\frac{1}{2\pi j}\oint_{C}\brak{\dfrac{x\brak{0}z^{n-1}}{\brak{1-z^{-1}}^2} + \dfrac{dz^{n-2}}{\brak{1-z^{-1}}^{3}}}\;dz  
\end{align}
For $R_1$ we can observe that the pole has been repeated twice.
\begin{align}
    R&=\frac{1}{\brak {m-1}!}\lim\limits_{z\to a}\frac{d^{m-1}}{dz^{m-1}}\brak {{(z-a)}^{m}f\brak z}\\
    R_1&=\frac{1}{\brak {1}!}\lim\limits_{z\to 1}\frac{d}{dz}\brak {{(z-1)}^{2}\frac{x\brak{0}z^{n+1}}{{(z-1)}^2}}\\
    &=x\brak{0}\brak{n+1}\lim\limits_{z\to 1}\brak{z^n}\\
    &=x\brak{0}\brak{n+1}\label{eq:1/ap/contour}
\end{align}
For $R_2$ we can observe that the pole has been repeated thrice.
\begin{align}
    R_2&=\frac{1}{\brak {2}!}\lim\limits_{z\to 1}\frac{d^{2}}{dz^{2}}\brak {{(z-1)}^{3}\frac{dz^{n+1}}{{(z-1)}^3}}\\
    &=\frac{d\brak{n+1}}{2}\lim\limits_{z\to 1}\frac{d}{dz}\brak{z^n}\\
    &=\frac{d\brak{n+1}\brak{n}}{2}\lim\limits_{z\to 1}\brak{z^{n-1}}\\
    &=\frac{d\brak{n}\brak{n+1}}{2}\label{eq:2/ap/contour} \\
    \implies R&= R_1 + R_2\label{eq:3/ap/contour}
\end{align}
Using \eqref{eq:1/ap/contour} and \eqref{eq:2/ap/contour} in \eqref{eq:3/ap/contour}
\begin{align}
   R &=x\brak{0}\brak{n+1}+\frac{d\brak{n}\brak{n+1}}{2}
\end{align}
Finally,
\begin{align}
    s\brak{n}&=x\brak{0}\brak{n+1}u\brak{n}+d\brak{\dfrac{{n\brak{n+1}}}{2}}u\brak{n}\\
    &=\dfrac{n+1}{2}\brak{2x\brak{0}+nd}u\brak{n}
\end{align}

\end{enumerate}
