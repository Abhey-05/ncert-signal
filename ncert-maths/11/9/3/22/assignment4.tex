\iffalse
\let\negmedspace\undefined
\let\negthickspace\undefined
\documentclass[journal,12pt,twocolumn]{IEEEtran}
\usepackage{cite}
\usepackage{amsmath,amssymb,amsfonts,amsthm}
\usepackage{algorithmic}
\usepackage{graphicx}
\usepackage{textcomp}
\usepackage{xcolor}
\usepackage{txfonts}
\usepackage{listings}
\usepackage{enumitem}
\usepackage{mathtools}
\usepackage{gensymb}
\usepackage{comment}
\usepackage[breaklinks=true]{hyperref}
\usepackage{tkz-euclide} 
\usepackage{listings}
\usepackage{gvv}                                        
\def\inputGnumericTable{}                                 
\usepackage[latin1]{inputenc}                                
\usepackage{color}                                            
\usepackage{array}                                            
\usepackage{longtable}                                       
\usepackage{calc}                                             
\usepackage{multirow}                                         
\usepackage{hhline}                                           
\usepackage{ifthen}                                           
\usepackage{lscape}

\newtheorem{theorem}{Theorem}[section]
\newtheorem{problem}{Problem}
\newtheorem{proposition}{Proposition}[section]
\newtheorem{lemma}{Lemma}[section]
\newtheorem{corollary}[theorem]{Corollary}
\newtheorem{example}{Example}[section]
\newtheorem{definition}[problem]{Definition}
\newcommand{\BEQA}{\begin{eqnarray}}
 \newcommand{\EEQA}{\end{eqnarray}}
\newcommand{\define}{\stackrel{\triangle}{=}}
\theoremstyle{remark}
\newtheorem{rem}{Remark}
\begin{document}
 \bibliographystyle{IEEEtran}
 \vspace{3cm}
 \title{\textbf{11.9.3.22}}
 \author{EE23BTECH11048-Ponugumati Venkata Chanakya$^{*}$% <-this % stops a space
 }
 \maketitle
 \newpage
 \bigskip
 \renewcommand{\thefigure}{\theenumi}
 \renewcommand{\thetable}{\theenumi}
 \textbf{QUESTION:}
If $p^{th},q^{th},r^{th} $ term of a GP are $a,b$ and $c$  respectively Prove that \\
\begin{align*}
    a^{q-r}b^{r-p}c^{p-q}=1
\end{align*}
\solution
\fi
\begin{align}
x(n)&=(x(0)d^n) u (n) \label{eq 11.9.3.22_1}\\
a&=x(p) = (x(0)d^p)\\
b&=x(q) = (x(0)d^q)\\
c&=x(r) = (x(0)d^r)\\
a^{q-r}b^{r-p}c^{p-q}&=x(0)^{q-r} d^{p(q-r)} x(0)^{r-p} d^{q(r-p)} x(0)^{p-q} d^{r(p-q)} \\
&= x(0)^{q-r+r-p+p-q} d^{p(q-r)+q(r-p)+r(p-q)}\\
&=x(0)^0 d^0\\
a^{q-r}b^{r-p}c^{p-q} &=1
\end{align}\

 \begin{table}[!ht]
    \centering
        \begin{tabular}{|c|c|c|} 
      \hline
\textbf{Variable}& \textbf{Description}& \textbf{Value}\\\hline
         $x(n)$& $n^{th}$ term of GP&none\\\hline
         $x(0)$& First term of GP&none\\\hline
          $d$&common ratio between the terms of GP&none\\\hline
          $x(p)$& a &$x(0)d^p$ \\ \hline
          $x(q)$& b &$x(0)d^q$ \\ \hline
          $x(r)$& c &$x(0)d^r$ \\ \hline
    \end{tabular}

    \caption{input parameters}
    \label{tab:11_9_3_22}
\end{table}

Taking Z-Transform:
\begin{enumerate}
    \item $\mathcal{Z}\{u(n)\}$
\begin{align}
    u(n) \system{Z} \frac{1}{1-z^{-1}} \{\abs{z} > 1\}\label{eq 11.9.3.22_9} 
\end{align}
    \item $\mathcal{Z}\{d^{n}u(n)\}$ 
\begin{align}
    nu(n) \system{Z} \frac{z^{-1}}{(1-dz^{-1})}\, \{\abs{z} > \abs{d}\} \label{eq 11.9.3.22_10}
    \end{align}
    Taking Z-Transform of \eqref{eq 11.9.3.22_1} using \eqref{eq 11.9.3.22_9}and \eqref{eq 11.9.3.22_10}
    \begin{align}
    X(z) &= \frac{x(0)}{1-dz^{-1}} \qquad |z| > |d|
     \end{align}
\end{enumerate}
%\end{document}
